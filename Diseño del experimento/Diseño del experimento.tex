\documentclass{article}
\usepackage[utf8]{inputenc}
\usepackage{parskip}

\title{III. DISEÑO DEL EXPERIMENTO}
\author{ }
\date{}

\begin{document}
%\twocolumn[l]
\setlength{\parskip}{0mm}
\maketitle
Implementaci{\'o}n  de un programa que nos pedir{\'a} ingresar orden del grafo, luego
se generar{\'a}n aleatoreamente todos los posibles grafos con n v{\'e}tices hasta encontrar uno que sea hamiltoniano el cual se podr{\'a} visualizar debido a la funci{\'o}n $plot$ y paquete $igraph$ utilizado. También podremos observar el ciclo hamiltoniano que estar{\'a} de un color color azul y las aristas que no pertenecen al ciclo de color verde.

\section{Funciones y objetos a utilizar}
\subsection{Plot} La funci{\'o}n plot es una funci{\'o}n gen{\'e}rica para la representación gr{\'a}fica de objetos en R. Los gr{\'a}ficos m{\'a}s sencillos que permite generar esta funci{\'o}n son nubes de puntos $(x, y)$.
\subsection{Grafos con igraph}
El paquete para Igraph, necesita que se le presente los datos de la matriz de adyacencia por parejas. Es decir, una matriz de doble entrada convencional (tambi{\'e}n llamada sociomatriz, tabla de confundido o tabla de concordancia) ha de pasarse al formato de igraph.
\subsection{nextPerm(V)}
Devuelve la siguiente permutaci{\'o}n del vector V usando ordenamiento lexicogr{\'a}fico de los valores en V. 
 Esta funci{\'o}n puede ser {\'u}til para generar permutaciones una a una, cuando el n{\'u}mero total de permutaciones es demasiado grande para almacenarlas todas en la memoria.
 \newline  
 \newline
\textbf{Este m{\'e}todo sigue los siguientes pasos:}
\begin{itemize}
    \item Encontrar el mayor valor de $x$ tal que $P[x] < P[x+1]$. Si dicho valor de $x$ no existe, entonces $P$ es la {\'u}ltima permutaci{\'o}n lexicogr{\'a}fica que puede ser concebida por medio del conjunto de elementos que la conforman.
    \item Encontrar el mayor valor de y tal que $P[x]<P[y]$. Si dicho valor de y no existe, entonces $P$ es la {\'u}ltima permutaci{\'o}n lexicogr{\'a}fica que puede ser concebida por medio del conjunto de elementos que la conforman.
    \item Intercambiar $P[x]$ por $P[y]$ y viceversa.
    \item Invertir los elementos desde $P[x+1]$ hasta $P[n]$.

\end{itemize}
\end{document}
